\documentclass[11pt,a4paper]{article}

% استفاده از بسته‌ی زیر الزامی نیست ولی با استفاده از آن می‌توانید لینکهای رنگی به مراجع خود داشته باشید. 
\usepackage[colorlinks]{hyperref}

\usepackage{xepersian}
\settextfont{XB Zar}
\setlatintextfont[Scale=1]{Linux Libertine}

\title{نمونه خروجی با استیل فارسی \lr{unsrt-fa} برای \lr{BibTeX} در زی‌پرشین}
\author{}\date{}
\begin{document}
\maketitle

مرجع \cite{Omidali82phdThesis} یک نمونه پروژه دکترا و مرجع\cite{Vahedi87} یک نمونه مقاله مجله فارسی است.
مرجع \cite{Baker02limits} یک نمونه مقاله انگلیسی است که در بین مراجع فارسی قرار گرفته است، مرجع \cite{Amintoosi87afzayesh}  یک نمونه  مقاله کنفرانس فارسی و
مرجع \cite{Pedram80osool} یک نمونه کتاب فارسی با ذکر مترجمان و ویراستاران فارسی است. مرجع \cite{Khalighi07MscThesis} یک نمونه پروژه کارشناسی ارشد انگلیسی و
\cite{Khalighi87xepersian} هم یک نمونه متفرقه  می‌باشند.

{\small
% در اینجا می‌توانید سبک‌های مختلف را گذاشته و تفاوت خروجی را ببینید
\bibliographystyle{ieeetr-fa}%{acm-fa}%{plain-fa}%{unsrt-fa}%

\bibliography{SomeReferences}
}

\end{document}
