\documentclass[11pt,a4paper]{article} 
% محمود امین‌طوسی، http://webpages.iust.ac.ir/mamintoosi

\usepackage{bibentry}
%\usepackage[linktocpage=true]{hyperref}%,colorlinks,pagebackref=true
\usepackage{hyperref}

\usepackage{xepersian}
\settextfont[Scale=1]{XB Zar}%{XB Zar}
\setlatintextfont[Scale=.95]{Linux Libertine}%{Arial}%
\setdigitfont{XB Zar}

\title{مثالی برای نمایش نحوهٔ استفاده از بستهٔ \lr{bibentry}}
\author{محمودامین‌طوسی}\date{}
\begin{document}
\nobibliography*
\maketitle
به صورت معمول در هنگام استفاده از دستور \lr{cite} شمارهٔ مراجع در متن ذکر می‌شود؛ ولی گاهی لازم است که خود مرجع نیز در محلی در متن ظاهر گردد. یک نمونه هنگامی است که در یک پروژه فرد می‌خواهد در یک بخش مطالبی را بیان نماید که قبلاً آنها را در قالب مقالاتی منتشر نموده است. 
با استفاده از بستهٔ \lr{bibentry} که همراه با \lr{natbib} ارائه می‌شود می‌توان به سادگی این کار را انجام داد.
با مطالعهٔ سورس این سند با نحوهٔ استفاده از آن آشنا خواهید شد. 

البته متاسفانه این بسته با \lr{backref} سازگار نیست؛ اخطاری دریافت نموده و کلمه صفحات در پایان مراجع داخل متن هم چاپ می‌شود (در حالیکه نیاز به آن نیست).

\section{مقدمه}

در این بخش روشهایی برای افزایش وضوح ناحیه‌ای یک تصویر با وضوح پایین با استفاده از تصاویر آموزشی با وضوح بالا معرفی خواهد شد\RTLfootnote{مطالب این بخش در قالب مقالات زیر منتشر شده است :\\
\begin{latin}\bibentry{Amintoosi09regional} \cite{Amintoosi09regional}
\end{latin}
\bibentry{Amintoosi87afzayesh} \cite{Amintoosi87afzayesh}
}.

\bibliographystyle{acm-fa}
\bibliography{MyReferences}


\end{document}